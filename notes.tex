\documentclass[a4paper]{article}
\usepackage[utf8]{inputenc}
\usepackage{amsthm}
\usepackage{amsfonts}
\usepackage{amssymb}
\usepackage{amsmath}
\usepackage{mathtools}
\usepackage[all]{xy}
\usepackage{color}
\usepackage{geometry}
\usepackage{hyperref}

\makeatletter
\renewcommand*\env@matrix[1][*\c@MaxMatrixCols c]{%
  \hskip -\arraycolsep
  \let\@ifnextchar\new@ifnextchar
  \array{#1}}
\makeatother


\SelectTips{eu}{}
\setlength{\fboxsep}{0pt}
\setlength\parskip{0.3em}
\setlength{\parindent}{0 pt}

\newcommand{\Aut}{\text{Aut}}
\newcommand{\Syl}{\text{Syl}}
\newcommand{\orb}{\text{orb}}
\newcommand{\stab}{\text{stab}}
\newcommand{\fix}{\text{fix}}
\newcommand{\Sym}{\text{Sym}}
\newcommand{\Alt}{\text{Alt}}
\newcommand{\supp}{\text{supp}}
\newcommand{\adj}{\text{adj}\ }
\newcommand{\lcm}{\text{lcm}\ }
\newcommand{\id}{\text{id}}
\newcommand{\im}{\text{im }}
\newcommand{\spanset}{\text{span}}
\newcommand{\rank}{\text{rank }}
\newcommand{\Mod}{\text{ mod }}

\theoremstyle{definition}

\newtheorem{defn}{Definition}[subsection]
\newtheorem{prop}[defn]{Proposition}
\newtheorem{thm}[defn]{Theorem}
\newtheorem{lemma}[defn]{Lemma}
\newtheorem{coro}[defn]{Corollary}
\newtheorem{example}[defn]{Example}
\newtheorem{exe}[defn]{Exercise}
\newtheorem*{remark}{Remark}
\newtheorem*{notation}{Notation}

\title{MA3K4 Introduction to group theory :: Lecture notes}
\author{Lecturer: Gareth Tracey}
\date{\today}

\begin{document}

\maketitle
\thispagestyle{empty}

\tableofcontents
\thispagestyle{empty}
\newpage
\setcounter{page}{1}

\begin{flushright}
\textit{Week 1, lecture 1 starts here}
\end{flushright}

\section{Introduction}
\begin{defn}
A \textit{group} is a pair $(G,\circ)$ where $G$ is a set and $\circ:G\times G\rightarrow G$ is a binary operation satisfying
\begin{enumerate}
\item Associativity: $(g\circ h)\circ k = g\circ (h\circ k) \ \forall g,h,k \in G$,
\item Identity: $\exists$ an element in $G$, denoted $1_G$, such that $1_G\circ g = g\circ 1_G = g \ \forall g\in G$,
\item Inverses: $\forall g\in G$, $\exists$ an element in $G$, denoted $g^{-1}$, such that $g\circ g^{-1} = g^{-1}\circ g = 1_G$.
\end{enumerate}
\end{defn}

\begin{remark}
Implicit in parts 1 and 2 of above definition are
\begin{enumerate}
\item An identity element in an associative binary operation is unique, justifying the notation and the `the' before `identity'
\item Similarly, inverses are unique in an associative binary operation, so we say \textit{the} inverse of $g$
\end{enumerate}
The number of elements in a group $(G,\circ)$ is called the order of $G$, denoted $|G|$.
\end{remark}

\begin{example}
Let $G=\mathbb Z$. Then
\begin{enumerate}
\item If we define $\circ:G\times G\rightarrow G$ by $g\circ h=g+h$ for $g,h\in \mathbb Z$ then we know $(G,\circ)$ is a group and $1_G=0,\ g^{-1}=-g \ \forall g\in G$.
\item For the same set, if we define $g\circ h=g\times h$ then $(G,\circ)$ is not a group for lack of inverses for $g\in \mathbb Z \backslash \{\pm 1\}$.
\end{enumerate}
\end{example}

\begin{remark}
\begin{enumerate}
\item You may have been given a fourth axiom, closure, in previously seen definitions of a group. The reason we omit that here is because it's implied by definition of binary operation.
\item If $(G,\circ)$ is a group, $\circ$ is often called the \textit{group operation}.
\item Given clear context, we will streamline our notation and simply write $G$ in place of $(G,\circ)$ and $gh$ in place of $g\circ h$.
\end{enumerate}
\end{remark}

\begin{defn}
Let $G$ be a group.
\begin{enumerate}
\item If $g,h\in G : gh=hg$ then $g$ and $h$ \textit{commute}.
\item If $g$ and $h$ commute $\forall g,h \in G$ then $G$ is \textit{abelian}.
\end{enumerate}
\end{defn}

\begin{example}
$(\mathbb Z,+)$ is abelian.
\end{example}

\begin{exe}[Commuting elements in groups]
Let $G$ be a group.
\begin{enumerate}
\item Suppose $g^2=1_G \ \forall g\in G$. Show that $G$ is abelian.
\begin{proof}
Note that this implies $\forall g,h\in G,\ (gh)^{-1}=gh$, but $(gh)^{-1}=h^{-1}g^{-1}=hg$, so $gh=hg$.
\end{proof}
\item Suppose $g^3=1_G \ \forall g\in G$. Show that $hgh^{-1}$ and $g$ commute $\forall g,h\in G$.
\begin{proof}
One has $g^2h=g^{-1}h^{-2}=(h^2g)^{-1}=h^2gh^2g\Rightarrow gh^2g=hg^2h\Rightarrow  hgh^2g=h^2g^2h$. Now consider $(gh)^{-1}$, which equals $h^2g^2$ but also $ghgh$. Hence $ghgh^{-1}=ghgh^2=h^2g^2h=hgh^2g=hgh^{-1}g$, as desired.
\end{proof}
\end{enumerate}
\end{exe}

Next, we are going to look at two infinite families of examples of groups: 1. Symmetric groups and 2. Linear groups.

\subsection{Symmetric group}
\begin{defn}
Let $X$ be a set, and define
\[
\Sym(X) = \{f:f:X\rightarrow X \text{ is a bijection}\}
\]
Define $\circ:\Sym(X)\times \Sym(X)\rightarrow \Sym(X)$ to be the usual composition of functions. Then $(\Sym(X),\circ)$ is a group, called the \textit{symmetric group} on $X$. An element of $\Sym(X)$ is called a \textit{permutation}. 
\end{defn}

\begin{remark}[Sanity check]
\begin{enumerate}
\item Associativity is clear by inheritance
\item $1_G=\id_X : x\mapsto x$
\item For $f\in \Sym(X)$, $x\in X$, choose a unique $y_x\in X$ such that $f(y_x)=x$. Define $g:X\rightarrow X$ by $g(x)=y_x$, then $g$ is a inverse for $f$.
\end{enumerate}
\end{remark}

We introduce cycle notation as a more compact way of writing permutations down.

\begin{flushright}
\textit{Week 1, lecture 2 starts here}
\end{flushright}

\begin{defn}[Cycle notation]
Let $X$ be a set.
\begin{enumerate}
\item Let $a_1,\ldots,a_n\in X$ be distinct. The permutation $f=(a_1,\ldots,a_n) \in\Sym(X)$ is defined to be $f(a_i)=a_{i+1}$ for $1\leq i\leq n-1$, $f(a_{n})=a_1$, and $f(b)=b$ for $b\not\in \{a_1,\ldots,a_n\}$. We call $f$ a \textit{cycle of length} $n$ (or an $n$-cycle).
\item Two cycles $(a_1,\ldots,a_r),\ (b_1,\ldots,b_s)$ are \textit{disjoint} if $\{a_1,\ldots,a_r\} \cap \{b_1,\ldots,b_s\}=\varnothing$.
\item The \textit{empty cycle}, written $()$, is the identity map which is also $1_{\Sym(X)}$.
\end{enumerate}
\end{defn}
\begin{remark}[Important points about cycles]
\begin{enumerate}
\item Perhaps a tautology, but the empty cycle is thought of as a cycle (of length 0).
\item Recall that the group operation is composition of functions. So $fg:X\rightarrow X$ means do $g$ first and then $f$. e.g. $X=\{1,2,3,4,5\}$, so $(3,4,1,2)(4,5)=(1,2,3,4,5)$.
\item Cycle notation is not unique in the following sense: two distinct $m$-tuples of elements in a set $X$ can represent the same cycle, e.g. $(1,2,3,4,5)=(3,4,5,1,2)$.
\end{enumerate}
\end{remark}

\begin{thm}
Let $X$ be a finite set. Then
\begin{enumerate}
\item $|\Sym(X)|=|X|!$,
\item Every element $F\in \Sym(X)$ can be written as product of disjoint cycles. Moreover, the decomposition is unique in the sense that if $F=f_1\cdots f_r=g_1\cdots g_s$ where $f_i,g_i$ are disjoint cycles of length $>1$, then $r=s$ and $\{f_1,\ldots,f_r\}=\{g_1,\ldots,g_r\}$.
\end{enumerate}
\end{thm}
\begin{proof}[Proof (nonexaminable)]
\begin{enumerate}
\item Write $X=\{x_1,\ldots,x_r\}$ where $n=|X|$ and define
\[
X(n):=\{(a_1,\ldots,a_n):a_i\in X, a_i\neq a_j \text{ for } i\neq j\}.
\]
Define a bijection $\theta:\Sym(X)\rightarrow X(n)$ by $\theta(f)=(f(x_1),\ldots,f(x_n))$. for $f\in \Sym(X)$, observe
\begin{enumerate}
\item $\theta$ is well-defined, since $f$ is a bijection, so $f(x_i)\neq f(x_j)$ for $i\neq j$.
\item In the same way, $\theta$ is injective. Indeed, if $\theta(f)=\theta(g)$ then $f(x_i)=g(x_i) \ \forall i$ by definition of $\theta$, so $f=g$.
\item If $(a_1,\ldots,a_n)\in X(n)$, then define $f:X\rightarrow X$ by $f(x_i)=a_i$ for $1\leq i\leq n$. Clearly, $f\in \Sym(X)$ and $\theta(f) = (a_1,\ldots,a_n)$, so $\theta$ is surjective.
\end{enumerate}
It follows that $|\Sym(X)|=|X(n)|=n!$.
\item Let $f\in\Sym(X)$. If $f=\id_X$ then $f=()$ so it's a cycle. Now suppose $f$ is not $\id_X$. Let $Y=\{x\in X:f(x)\neq x\}$. Note that since $|\Sym(X)|$ is finite by 1., $\exists n\in \mathbb N$ such that $f^n=\id_X$.

In particular, if we fix $a_1\in Y$, then we may define $m_1:=\min\{m\in \mathbb N:f^m(a_1)=a_1\}$ since the set is nonempty. Now, for $2\leq i\leq m_1$, define $a_i:=f(a_{i-1})$. If $Y=\{a_1,\ldots,a_{m_1}\}$, then by definition of cycle, one has $f=(a_1,\ldots,a_m)$.

Now suppose $Y\backslash \{a_1,\ldots,a_{m_1}\}\neq \varnothing$. Choose $a_{m_1+1} \in Y\backslash \{a_1,\ldots,a_{m_1}\}$, and define $m_2:=\min\{m\in \mathbb N:f^m(a_{m_1+1})=a_{m_1+1}\}$. For $m_1+2\leq i\leq m_2$, again define $a_i:=f(a_{i-1})$, then if $Y=\{a_1,\ldots,a_{m_1},a_{m_1+1},\ldots,a_{m_2}\}$, one has $f=(a_1,\ldots,a_m)(a_{m+1},\ldots,a_{m_2})$. If not, we continue inductively. Since $X$ is finite, this must terminate, and when it does $f$ will be a product of disjoint cycles. The uniqueness follows from the algorithm immediately.
\end{enumerate}
\end{proof}

\subsection{Linear group}
\begin{defn}
$F$ is a field and $n\in \mathbb N$. We define
\[
GL_n(F) := \{A:A\text{ an invertible }n\times n \text{ matrix over } F\},
\]
a group with matrix multiplication as operation. This is called \textit{general linear group} of dimension $n$ over $F$.
\end{defn}
\begin{flushright}
\textit{Week 1, lecture 3 starts here}
\end{flushright}

\begin{remark}[Useful things from Algebra I, II for studying general linear groups]
\begin{enumerate}
\item Each field $F$ has an additive and multiplicative identity $0_F$ and $1_F$. Given clear context, they will be denoted simply 0 and 1 respectively.
\item An $n\times n$ matrix $A$ over $F$ is invertible iff $\det A\neq 0$ iff rows (or columns) of $A$ are linearly independent.
\item If $F$ is a finite field, then $|F|=p^f$ for some prime $p$ and $f\in\mathbb N$. Moreover, for each prime $p$ and each $f\in\mathbb N$, $\exists!$ a field (up to isomorphism) $F:|F|=p^f$. $p$ is called the \textit{characteristic} of $F$, and satisfies that $p\alpha=0 \ \forall \alpha\in F$.
\item If $F$ is a field then $F^{\times}:=F\backslash \{0\}$ is a group with multiplication as group operation inherited from $F$.
\end{enumerate}
\end{remark}
\begin{exe}
\begin{enumerate}
\item Let $X$ be a set. Show that $\Sym(X)$ is abelian iff $|X|\leq 2$.
\item Let $F$ be a field. Show that $GL_n(F)$ is abelian iff $n=1$.
\end{enumerate}
\end{exe}
\begin{thm}
\label{thm:orderofGLnF}
Let $F$ be a finite field with $|F|=q$. Then $\displaystyle |GL_n(F)|=q^{\binom{n}{2}} \prod_{i=1}^n (q^i-1)$.
\end{thm}
\begin{proof}[Proof (nonexaminable)]
See sheet 1. 
\end{proof}

\subsection{Order of elements}
\begin{defn}
The \textit{order} of $g\in G$, denoted $|g|$, is defined $|g|:=\min \{n\in\mathbb N:g^n= 1_G\}$. If the set is $\varnothing$ then $|g|:=\infty$.
\end{defn}
\begin{example}
\begin{enumerate}
\item Let $X$ be a set and let $f=(a_1,\ldots,a_m)\in\Sym(X)$. Then $|f|=m$.
\item Let $F$ be a finite field of order $p^f$ where $p$ prime, $G=GL_2(F)$, and $\alpha,\beta\in F^{\times}$. Observe that
\[
\begin{pmatrix}
1 & \alpha \\ 0 & 1
\end{pmatrix}
\begin{pmatrix}
1 & \beta \\ 0 & 1
\end{pmatrix}=
\begin{pmatrix}
1 & \alpha+\beta \\ 0 & 1.
\end{pmatrix}
\]
So if $g=\begin{pmatrix}
1 & \alpha \\ 0 & 1
\end{pmatrix}$ then $g^n=\begin{pmatrix}
1 & n\alpha \\ 0 & 1
\end{pmatrix}$, so $|g|\mid p$ (we'll see later about this implication), so $|g|=p$.

Also,
\[
g^n=\begin{pmatrix}
\alpha & 0 \\ 0 & \beta
\end{pmatrix}^n=
\begin{pmatrix}
\alpha^n & 0 \\ 0 & \beta^n .
\end{pmatrix}
\]
So $|g|=\lcm (m,k)$ where $m=|\alpha|$ and $k=|\beta|$ as elements of $F^\times$.
\end{enumerate}
\end{example}
\begin{remark}
\begin{enumerate}
\item For $g\in G$, $(g^n)^{-1}=\left(g^{-1}\right)^n$, so we write $g^{-n}:=\left(g^{-1}\right)^n$. In particular, $\left|g^{-1}\right|=|g|$.
\item If $g\in G,\ n=|g|$ and $n\mid l$, then $g^l=1$.
\end{enumerate}
\end{remark}
\begin{lemma}
\label{lemma:orderlem}
Let $a,b\in G$ of finite order. Then
\begin{enumerate}
\item If $l\in \mathbb N$, then $a^l=1$ iff $|a|\mid l$.
\item Let $m\in \mathbb N$, then $\displaystyle |a^m|=\frac{|a|}{\gcd (|a|,m)}$.
\item If $a,b$ commute then $|ab|\mid \lcm (|a|,|b|)$.
\item If $a,b$ commute and $a^i=b^j \ \forall i,j\in \mathbb N$ only when they are both 1 (i.e. $\langle a\rangle \cap \langle b\rangle =\{1\}$) then $|ab|= \lcm (|a|,|b|)$.
\end{enumerate}
\end{lemma}
\begin{proof}
\begin{enumerate}
\item $\Leftarrow$ is mentioned. $\Rightarrow$: suppose $a^l=1$. By Euclidean division, we can write $l=q|a|+r$ for some $r\in[0,|a|)$. Then $1=a^l=a^{q|a|+r}=a^r$, which contradicts minimality of $|a|$.
\item Suppose first that $m\mid |a|$. Then one can write $|a|=ms$, so $a^{ml}=1\Leftrightarrow |a| \mid ml$ by 1 $\displaystyle \Leftrightarrow \frac{|a|}{m}\mid l$. Hence the least positive integer $l:a^{ml}=1$ is $\displaystyle \frac{|a|}{m}$.

Now let $k=\gcd (|a|,m)$. We write $m=ks$, then $a^{m\frac{|a|}{k}}=a^{|a|s}=1$, and by 1 one has $|a^m| \mid \frac{|a|}{k}$. To complete the proof it suffices to show that $\frac{|a|}{k}\leq |a^m|$.
\begin{flushright}
\textit{Week 2, lecture 1 starts here}
\end{flushright}
By Bézout's lemma, $\exists s,t\in\mathbb Z:k=s|a|+tm$, so $a^k=a^{s|a|+tm}=(a^{|a|})^s a^{tm}=a^{tm}$. Then $a^{tm|a^m|}=((a^m)^{|a^m|}))^t=1^t=1$. This implies $|a^{tm}|\mid a^m$ by 1. So $\frac{|a|}{k}=|a^k|=|a^tm|\mid |a^m|$.
\item Let $l:=\lcm(|a|,|b|)$. Then $(ab)^l=a^lb^l=1\times 1=1$, so by 1. $|ab|\mid l$.
\item Let $k:=|ab|$. Then $k\mid l$, but also, $1=(ab)^k=a^kb^k$ so $a^k=\left(b^{-1}\right)^k$ and by assumption both sides are 1. So $|a|,|b|\mid k$, so $l\mid k$, hence $k=l$.
\end{enumerate}
\end{proof}
\begin{exe}
\label{exe:conjaugatesordersequal}
\begin{enumerate}
\item Let $h,g\in G$. Show that $\left|hgh^{-1}\right|=|g|$.
\item Let $l,m,n>2\in\mathbb N$. Show that $\exists G$ with $a,b\in G:|a|=l,\ |b|=m,\ |ab|=n$. Also:
\begin{enumerate}
\item Show that $G$ can be finite.
\item Show that one can replace $l,m,n>2$ by $l,m,n>1$.
\end{enumerate}
Key hint: A $2\times 2$ matrix over $\mathbb C$ with distinct eigenvalues is diagonalisable. Now exploit result of 1st exercise.
\end{enumerate}
\end{exe}
\subsection{Subgroup and coset}
\begin{defn}
A nonempty $H\subseteq G$ is a \textit{subgroup} of $G$, denoted $H\leq G$, if
\begin{enumerate}
\item $1_G\in H$
\item $h\in H\Rightarrow h^{-1}\in H$
\item $h_1,h_2\in H \Rightarrow h_1h_2\in H$
\end{enumerate}
\end{defn}
\begin{defn}
For a group $G$ and $g\in G$, define $\langle g\rangle:=\{g^n:n\in\mathbb Z\}$ which is called the \textit{cyclic subgroup of} $G$ \textit{generated by} $g$. If $G=\langle g\rangle$ then $G$ is \textit{cyclic} and $g$ is a \textit{generator} for $G$.
\end{defn}
\begin{lemma}
\label{lemma:sgptest}
$H\subseteq G$ where $H$ nonempty. $H\leq G \Leftrightarrow h_1h_2\in H\Rightarrow h_1h_2^{-1}\in H$
\end{lemma}
\begin{proof}
\begin{itemize}
\item[$\Rightarrow$] $h_1,h_2\in H\Rightarrow h_2^{-1}\in H\Rightarrow h_1h_2^{-1}\in H$.
\item[$\Leftarrow$] \begin{enumerate}
\item $H\neq \varnothing\Rightarrow h\in H\Rightarrow hh^{-1}\in H\Rightarrow 1_G\in H$
\item $h\in H\Rightarrow 1_Gh^{-1}=h^{-1}\in H$
\item $h_1,h_2\in H\Rightarrow h_2^{-1}\in H\Rightarrow h_1(h_2^{-1})^{-1}h_1h_2\in H$
\end{enumerate}
\end{itemize}
\end{proof}
\begin{example}
Let $G=GL_2(F)$ and
\[
H=\left\{\begin{pmatrix}
  \alpha & 0 \\ 0 & \beta
\end{pmatrix}:\alpha,\beta\in F^{\times}\right\} \subseteq G. \qquad\text{sometimes called diagonal subgroup}
\]
We want to show this is indeed a subgroup. Let $h_i=\begin{pmatrix}
  \alpha_i & 0 \\ 0 & \beta_i
\end{pmatrix}\in H$ where $i=1,2$. Then
\[
h_1h_2=\begin{pmatrix}
  \alpha_1 & 0 \\ 0 & \beta_2
\end{pmatrix}\begin{pmatrix}
  \alpha_2^{-1} & 0 \\ 0 & \beta_2^{-1}
\end{pmatrix}=\begin{pmatrix}
  \alpha_1 \alpha_2^{-1} & 0 \\ 0 & \beta_1\beta_2^{-1}
\end{pmatrix}\in H .
\]
\end{example}
\begin{defn}
Let $A\subseteq G$ be nonempty. The \textit{subgroup of} $G$ \textit{generated by} $A$, denoted $\langle A\rangle$, is
\[
\{a_1^{\varepsilon_1}\cdots a_m^{\varepsilon_m}:m\in\mathbb N,\ a_i\in A,\ \varepsilon_i=\{\pm 1\}\}.
\]
\end{defn}
\begin{notation}
If $A=\{g_1,\ldots,g_t\}$ then we often write $\langle A\rangle$ as $\langle g_1,\ldots,g_t\rangle$.
\end{notation}

\begin{flushright}
\textit{Week 2, lecture 2 starts here}
\end{flushright}

\begin{exe}
Let $G$ be a group and $A\subseteq$ nonempty.
\begin{enumerate}
\item Use Lemma \ref{lemma:sgptest} to show that $\langle A\rangle$ is indeed a subgroup of $G$.
\item Write $A=\{g_1,\ldots,g_s\}$ and suppose $g_ig_j=g_jg_i \ \forall i,j=1,\ldots,s$. Show that $|\langle A\rangle|\leq \prod_{i=1}^s |g_i|$.
\item Suppose $g^p=1 \ \forall g\in G$ and $G=\langle x,y\rangle$ for some $x,y\in G$.
\begin{enumerate}
\item Show that if $p=2,\ |G|\leq 4$.
\item Show that if $p=3,\ |G|\leq 3^4$.
\item Fields-medal-worth: If $p=5$, is $G$ finite?
\end{enumerate}
\end{enumerate}
\end{exe}

\begin{defn}
The \textit{left coset} of $H\leq G$ with respect to $g\in G$ is the set $gH:=\{gh:h\in H\}.$ The \textit{right coset} is defined similarly.
\end{defn}
$gH$ is not a subgroup unless $g\in H$ since in general the identity is not there.
\begin{lemma}
Let $H\leq G$ and $g,k\in G$. The following are equivalent:
\begin{enumerate}
\item $k\in gH$
\item $kH=gH$
\item $g^{-1}k\in H$
\end{enumerate}
\end{lemma}
\begin{proof}
First note that if $h\in H$ then $hH=H$ by virtue of the fact $H\leq G$.

Now $k\in gH\Rightarrow k=gh$ for some $h\in H \Rightarrow kH=ghH=gH$, so 1 implies 2. The other two implications are almost identical.
\end{proof}
\begin{lemma}
Let $H\leq G$. For $g_1,g_2\in G$, say that $g_1\sim_H g_2\Leftrightarrow g_1H=g_2H$. Then $\sim_H$ is an equivalence relation.
\end{lemma}
\begin{proof}
The three conditions reflexivity, symmetry and transitivity follow immediately from definition.
\end{proof}
\begin{coro}
\label{coro:cosetspartitiongp}
Let $H\leq G$.
\begin{enumerate}
\item If $g_1,g_2\in G$, then either $g_1H=g_2H$ or $g_1H\cap g_2 H=\varnothing$.
\item The set $\{gH:g\in G\}$ of left cosets is a partition of $G$, i.e. if $g_iH$ for $i\in I$ are distinct left cosets of $H$ in $G$ then
\[
G=\bigsqcup_{i\in I} g_i H.
\]
\end{enumerate}
\end{coro}
\begin{proof}
$\{gH:g\in G\}$ is precisely the set of equivalence classes under $\sim_H$, so the results follow immediately.
\end{proof}
\begin{thm}[Lagrange's]
Let $G$ be a finite group and $H\leq G$. Then $|H|\mid |G|$.
\end{thm}
\begin{proof}
Let $g_1H,\ldots,g_tH$ be distinct left cosets of $H$ in $G$. By Corollary \ref{coro:cosetspartitiongp},
\[
|G|=\left|\bigsqcup_{i=1}^t g_i H\right|=\sum_{i=1}^t |g_iH|,
\]
and one also has $|gH|=|H| \ \forall g\in G$ since $gH\rightarrow H$ defined by $gh\mapsto h$ is a bijection. Hence $|G|=t|H|$.
\end{proof}

\begin{defn}
\begin{enumerate}
\item As in the context of above, we write $G/H:=\{gH:g\in G\}$.
\item $|G/H|$ is called \textit{index} of $H$ in $G$, denoted $|G:H|$. By Lagrange's theorem if $G$ is finite then $|G:H|=\frac{|G|}{|H|}$.
\end{enumerate}
\end{defn}
\begin{coro}
If $G$ is finite and $g\in G$, then $|g|\mid |G|$.
\end{coro}
\begin{proof}
This follows from the fact $|\langle g\rangle |=|g|$ and Lagrange's theorem.
\end{proof}

\subsection{Normal subgroup and quotient group}
In general $G/H$ is not a group, which is the motivation of this section.
\begin{lemma}
Let $H\leq G,\ g\in G$. Then $gHg^{-1}=\left\{ghg^{-1}:h\in H\right\} \leq G$.
\end{lemma}
\begin{proof}
We use Lemma \ref{lemma:sgptest}. Clearly $gHg^{-1}\neq \varnothing$ since $1_G\in gHg^{-1}$. Now let $x=gh_1g^{-1},\ y=gh_2g^{-1}$ where $h_1,h_2\in H$. Note that $h_1h_2\in H$ since $H\leq G$. Then $y^{-1}=gh_2^{-1}g^{-1}$ so
\[
xy^{-1}=gh_1g^{-1}gh_2^{-1}g^{-1}=gh_1h_2^{-1}g^{-1} \in gHg^{-1}.
\]
\end{proof}
\begin{defn}
\begin{enumerate}
\item $H\leq G$ is \textit{normal} in $G$ if $gHg^{-1}=H \ \forall h\in H$, denoted $N\unlhd G$.
\item The \textit{normaliser} of $H\leq G$ is defined as
\[
N_G(H):=\{g\in G:gHg^{-1}=H\}.
\]
\end{enumerate}
\end{defn}
\begin{exe}
\begin{enumerate}
\item If $H\leq G$, show that $N_G(H)\leq G$.
\item $\{1_G\},G$ are always normal.
\end{enumerate}
\end{exe}

\begin{flushright}
\textit{Week 2, lecture 3 starts here}
\end{flushright}

\begin{defn}
$G$ is \textit{simple} if $\{1_G\}$ and $G$ are the only normal subgroups of $G$.
\end{defn}
\begin{example}
\begin{itemize}
\item $\mathbb Z/p\mathbb Z$ for any prime $p$ (by Lagrange's)
\item $A_n$ for $n\geq 5$
\end{itemize}
\end{example}
\begin{notation}
$AB:=\{ab:a\in A,b\in B\}$ where $A,B\subseteq G$. It's a subset but not a subgroup of $G$ in general, even if $A,B\leq G$.
\end{notation}
\begin{lemma}
Let $N\unlhd G$ and $g,h\in G$. Then $(gN)(hN)=ghN$.
\end{lemma}
\begin{proof}
\begin{itemize}
\item[$\subseteq$:] Let $x=gn_1\in gN,\ y=hn_2\in hN$ where $n_{1,2}\in N$. Then
\[
xy=gn_1hn_2=ghh^{-1}n_1hn_2 \in ghN
\]
since $h^{-1}n_1h\in N$ by definition of a normal subgroup.
\item[$\supseteq$:] Let $x=ghn\in ghN$ where $n\in N$. Then
\[
x=(g1_G)(hn)\in (gN)(hN).
\]
\end{itemize}
\end{proof}
\begin{defn}
Let $N\unlhd G$.
\begin{enumerate}
\item The \textit{natural binary operation} on $G/N$ is $\circ:G/N\times G/N\rightarrow G/N$ given by $(gN)\circ(hN)=ghN$.
\item $(G/N,\circ)$ is a group, called the \textit{quotient of} $G$ \textit{by} $N$.
\end{enumerate}
\end{defn}
Checking this is indeed a group is left as an exercise.

\subsection{Homomorphisms}
\begin{defn}
\begin{enumerate}
\item A map $\theta:G\rightarrow H$ is a \textit{homomorphism} if $\theta(g_1g_2)=\theta(g_1)\theta(g_2) \ \forall \theta_{1,2}\in G$.
\item A bijective homomorphism is an \textit{isomorphism}. If for $G,H,\ \exists\theta:G\rightarrow H$ an isomorphism, then $G$ and $H$ are \textit{isomorphic}, denoted $G\cong H$.
\item Let $\theta:G\rightarrow H$ be a homomorphism. The \textit{kernel} of $\theta$, denoted $\ker \theta$, is defined to be $\{g\in G:\theta (g)=1_H\}$, which is a subgroup of $G$. The \textit{image} of $\theta$, denoted $\im \theta$, is defined to be $\{\theta(g):g\in G\}$.
\end{enumerate}
\end{defn}
\begin{example}
Let $F$ be a field, $G=GL_n(F)$ and $H=F^\times$. Then $\det G\rightarrow H$ is a (surjective) homomorphism, since $\det AB=\det A\det B \ \forall A,B\in GL_n(F)$. Also
\[
\ker \det = \{A\in GL_n(F):\det A=1_F\}=: SL_n(F).
\]
\end{example}

\begin{thm}[1st isomorphism theorem]
Let $\theta:G\rightarrow H$ be an homomorphism. Then
\begin{enumerate}
\item $\ker\theta\unlhd G$.
\item $\im \theta\leq H$.
\item $G/\ker\theta\cong\im\theta$.
\end{enumerate}
\end{thm}

\begin{thm}[2nd isomorphism theorem]
Let $H\leq G$ and $N\unlhd G$. Then
\begin{enumerate}
\item $HN=NH\leq G$.
\item $H\cap N\unlhd H$.
\item $HN/N\cong H/(H\cap N)$.
\end{enumerate}
\end{thm}

\begin{thm}[3rd isomorphism theorem]
Let $N,K\unlhd G : N\leq K$. Then
\[
(G/N)/(K/N)\cong G/K.
\]
\end{thm}

\begin{thm}[Correspondence (or 4th isomorphism) theorem]
Let $N\unlhd G$. Then the map
\[
f:\{J:N\leq J\leq G\} \rightarrow \{X:X\leq G/N\}
\]
given by
\[
J\mapsto J/N
\]
is a bijection.
\end{thm}
\begin{proof}
Let $A:=\{J:N\leq J\leq G\}$ and $B:=\{X:X\leq G/N\}$. Clearly $J/N\leq G/N$.

Suppose $J_{1,2}\in A$ and $f(J_1)=f(J_2)$, and let $x\in J_1$. Then
\[
xN\in f(J_1)=f(J_2)=J_2/N,
\]
so $xN=yN$ for some $y\in J_2$. Since $x\in xN$, $x=yn\in J_2$ for some $n\in N$. It follows that $J_1\subseteq J_2$, and symmetrically $J_2\subseteq J_1$. Hence $f$ is injective.

Let $X\in B$ and set $Y=\{y\in G:yN\in X\}$. One can see that $Y\leq G$ since $y_{1,2}N\in X\Rightarrow (y_1N)(y_2N)^{-1}\in X\Rightarrow y_1y_2^{-1}N \in X$, so $y_1y_2^{-1}\in Y$ by definition, hence $Y\leq G$. Since $N\leq Y$ ($nN=N=1_{G/N}\in X \ \forall n\in N$) one has $y\in A$. Since $f(Y)=X,\ f$ is surjective.
\end{proof}

\begin{flushright}
\textit{Week 3, lecture 1 starts here}
\end{flushright}

\section{Group action}
\subsection{Permutation groups}
\begin{defn}
Let $X$ be a set. $G\leq\Sym(X)$ is called a \textit{permutation group} on $X$.
\end{defn}

\begin{defn}
\begin{enumerate}
\item Let $g\in\Sym(X)$. The \textit{support} of $g$ is defined
\[
\supp(g):=\{x\in X:g(x)\neq x\} \subseteq X.
\]
\item Let $G\leq \Sym(X)$. The \textit{support} of $G$ is defined
\[
\supp(G):=\{x\in X:g(x)\neq x\text{ for some }g\in G\} \subseteq X.
\]
\end{enumerate}
\end{defn}
\begin{example}
\begin{enumerate}
\item $\supp(\Sym(X))=X$.
\item $\supp(\{1_G\})=\varnothing$.
\item $X=\{1,2,3,4,5,6\}$ and $g=(1,5,6)$. Then $\supp(g)=\{1,5,6\}$.
\item $X=\{1,2,3,4,5\}$ and $g=(1,2)(3,5)$. Then $\supp(g)=\{1,2,3,5\}$.
\end{enumerate}
\end{example}
\begin{remark}
As the above examples show, one can read off the support of $g\in\Sym(X)$ from its decomposition as a product of disjoint cycles. More precisely, if $f\in\Sym(X)$, $f=f_1\ldots f_m$ is such decomposition where $f_i=\left(a_{i_1},\ldots,a_{i_{t_i}}\right)$. Then
\[
\supp(f)=\{a_{i_j} : 1\leq i\leq m, \ 1\leq j \leq t_i\}.
\]
\end{remark}
\begin{exe}
Let $H,G\leq \Sym(X)$.
\begin{enumerate}
\item Show that $H\leq G\Rightarrow\supp(H)\subseteq\supp(G)$.
\item Deduce that $\supp(H)\cap\supp(G)\Rightarrow H\cap G =\{1_{\Sym(X)}\}$.
\item Is the converse of above true?

No, counterexample: $X=\{1,2,3\},\ G=\langle(1,2)\rangle,\ H=\langle(2,3)\rangle$.
\item What if $gh=hg \ \forall g\in G,h\in H$?
\end{enumerate}
\end{exe}

\begin{thm}
\label{thm:disjointcycle}
\begin{enumerate}
\item Disjoint cycles commute.
\item Let $f\in\Sym(X)$ and $f=f_1\ldots f_m$ as a product of disjoint cycles $f_i$. If $m=1$ then $|f|$ is length of $f_1$. If $m\geq 2$ then $|f|=\lcm (|f_1|,\ldots,|f_m|)$.
\item If $f=(a_1,\ldots,f_r)\in\Sym(X)$ is a cycle and $g\in\Sym(X)$, then $^gf:=gfg^{-1}=\left(g(a_1),\ldots,g(a_r)\right)$.
\end{enumerate}
\end{thm}

\begin{proof}[Proof (nonexaminable)]
\begin{enumerate}
\item Let $f=(a_1,\ldots,a_r),\ g=(b_1,\ldots,b_s)$ be disjoint cycles. One needs to prove $(f\circ g)(x)=(g\circ f)(x) \ \forall x\in X$.

Suppose $x\in\{a_1,\ldots,a_r\}$, which implies $x\neq b_i$ by assumption. So $g(x)=x$ by definition of cycles, hence $f(g(x))=f(x)$. Also, again by definition, $f(x)\in\{a_1,\ldots,a_r\}$, so $f(x)\neq b_i$, hence $g(f(x))=f(x)$. The argument for case $x\notin \{a_1,\ldots,a_r\}$ is symmetric.
\item The case $m=1$ is seen before in section 1.3. We prove the claim by induction on $m$. Suppose $m\geq 2$ and all precedents are true. Let $g=f_1\ldots f_{m-1}$. We now need three things to finish the proof:
\begin{enumerate}
\item Write $f_i=(a_{i_1},\ldots,a_{i_{t_i}})$. Then $\supp(g)=\{a_{i_j}:1\leq i\leq m-1,\ 1\leq j\leq t_i\}$ and $\supp(f_m)=\{a_{m_j}:1\leq j\leq t_m\}$. By assumption $\supp(g)\cap\supp(f_m)=\varnothing$, so $\langle g\rangle \cap \langle f_m\rangle=\{1_{\Sym(X)}\}$ by exercise above.
\item $g$ and $f_m$ commute by 1.
\item $|g|=\lcm(|f_1|,\ldots,|f_{m-1}|)$ by inductive hypothesis.
\end{enumerate}
By Lemma \ref{lemma:orderlem}.4 one has the desired.
\item Let $b_i:=g(a_i)$ and observe that $(gfg^{-1})(b_i)=gfg^{-1}(g(a_i))=g(f(a_i))=g(a_{i+1})=b_{i+1}$. Now let $x\in X\backslash\{b_1,\ldots,b_m\}$. Then $g^{-1}(x)\in X\backslash\{g^{-1}(b_1),\ldots,g^{-1}(b_m)\}$ since $g$ is a bijection, i.e. $g^{-1}(x)\in X\backslash\{a_1,\ldots,a_m\}$, so $f(g^{-1}(x))=g^{-1}(x)$, and $gfg^{-1}(x)=g(g^{-1}(x))=x$.
\end{enumerate}
\end{proof}

\begin{flushright}
\textit{Week 3, lecture 2 starts here}
\end{flushright}
Recall that a subgroup of $G$ generated by a nonempty $A\subseteq G$ is defined to be
\[
\langle A\rangle :=\left\{ a_1^{\varepsilon_1}\cdots a_m^{\varepsilon_m}:m\in\mathbb N,\ \varepsilon_i\in\{\pm 1\},\ a_i\in A \right\}.
\]
\begin{exe}
\label{exe:checkgeneratingset}
Let $A\subseteq G$ be nonempty.
\begin{enumerate}
\item Show that
\[
\langle A\rangle = \bigcap_{A\subseteq H\leq G} H.
\]
In particular, if $H\leq G$ and $A\subseteq H$ then $\langle A\rangle\leq H$.
\item Recall that given $H\leq G,\ N_G(H):=\{g\in G:gHg^{-1}=H\}$. Suppose $g\in G$ and $gag^{-1}\in\langle A\rangle \ \forall a\in A$. Show that $g\in N_G(\langle A\rangle).$ (One only needs to check element in generating set instead of the whole subgroup for normaliser.)
\end{enumerate}
\end{exe}
\begin{defn}
\label{defn:dihedralgp}
Let $n\in\mathbb N,\ n\geq 3$ and set $X:=\{1,\ldots,n\}$. Define $\sigma,\tau\in\Sym(X)$ by $\sigma:=(1,2,\ldots,n)$ and $\tau=\prod_{i=1}^{\lfloor n/2\rfloor} (i,n-i+1)=(1,n)(2,n-1)\cdots$. The \textit{dihedral group of order} $2n$ is the permutation group on $X$ defined by $D_{2n}:=\langle\sigma,\tau\rangle.$
\end{defn}
This is the rigorous (algebraic) definition of $D_{2n}$, but it can also be thought of group of symmetries of a regular $n$-gon.
\begin{example}
\begin{enumerate}
\item $n=8,\ \sigma=(1,2,3,4,5,6,7,8),\ \tau=(1,8)(2,7)(3,6)(4,5)$.
\item $n=7,\ \sigma=(1,2,3,4,5,6,7),\ \tau=(1,7)(2,6)(3,5)$.
\end{enumerate}
\end{example}

\begin{thm}
Let $n\in\mathbb N,\ n\geq 3$.
\begin{enumerate}
\item $|D_{2n}|=2n$.
\item $N:=\langle\sigma\rangle \unlhd D_{2n}$ and $|N|=n$.
\end{enumerate}
\end{thm}
\begin{proof}
\begin{enumerate}
\item See sheet 2.
\item First note that $\tau\sigma\tau^{-1}=(\tau(1),\ldots,\tau(n))=(n,n-1,\ldots,1)=\sigma^{-1}$ by Theorem \ref{thm:disjointcycle}.3 and definition of $\tau$. Also clearly $\sigma\sigma\sigma^{-1}=\sigma$. Now if $A:=\{\sigma\}$ then we have shown $^\tau \sigma,^\sigma \sigma \in\langle A\rangle$, so by Exercise \ref{exe:checkgeneratingset}.2, $\tau,\sigma\in N_{D_{2n}}(\langle A\rangle)$. Hence $\langle\{\tau,\sigma\}\rangle=D_{2n}\subseteq N_{D_{2n}}(\langle A\rangle)$, i.e. $\langle A\rangle \unlhd D_{2n}$. Also $|N|=|\langle \sigma\rangle|=|\sigma|=n$.
\end{enumerate}
\end{proof}

\begin{defn}
Let $X$ be a finite set.
\begin{enumerate}
\item Let $f\in\Sym(X)$ and write $f=f_1\cdots f_m$ as product of disjoint cycles. $f$ is \textit{even} if the number of cycles of even length in $\{f_1,\ldots,f_m\}$ is even. Otherwise $f$ is \textit{odd}.
\item The \textit{alternating group on} $X$, denoted $\Alt(X)$, is defined $\{f:f\in\Sym(X)\text{ even}\}$.
\end{enumerate}
\end{defn}
\begin{example}
$(1,2,3,4)\in S_4$ is odd, $(1,2)(3,4,5)\in S_5$ is odd, $(1,2)(3,4,5,6)\in S_6$ is even.
\end{example}

\begin{prop}
$\Alt(X)\leq \Sym(X)$ and $[\Sym(X):\Alt(X)]=2$, i.e. $|\Alt(X)|=\frac{|X|!}{2}$.
\end{prop}
\begin{proof}
See sheet 2.
\end{proof}

\begin{prop}
If $X,Y$ are finite sets with $|X|=|Y|$, then $\Sym(X)\cong\Sym(Y)$.
\end{prop}
\begin{proof}
Let $\beta:X\rightarrow Y$ be a bijection. Define $\theta:\Sym(X)\rightarrow\Sym(Y)$ by $f\mapsto \beta f\beta^{-1}$. It's then clear that $\theta$ is an isomorphism.
\end{proof}

\begin{flushright}
\textit{Week 3, lecture 3 starts here}
\end{flushright}

Recall that if $G=\langle B\rangle,\ H=\langle A\rangle$, then $H\unlhd G\Leftrightarrow bab^{-1}\in H \ \forall a\in A, b\in B$.

\subsection{Group actions}
\begin{defn}
Let $G$ be a group and $X$ a set. An \textit{action} of $G$ on $X$ is a map $\cdot:G\times X\rightarrow X$ such that
\begin{enumerate}
\item $1_G\cdot x=x \quad \forall x\in X$
\item $(gh)\cdot x = g\cdot (h\cdot x) \quad \forall g,h\in G,\ x\in X$
\end{enumerate}
We say $G$ \textit{acts on} $X$ and $X$ is a a $G$\textit{-set}.
\end{defn}

\begin{example}
\label{example:gpaction}
\begin{enumerate}
\item The action of $G$ on itself by left multiplication: let $X:=G$ and define $\cdot :G\times X\rightarrow X$ by $g\cdot x:= gx,\ g\in G,\ x\in X$. Note that by definition of a group,
\begin{enumerate}
\item $1_G\cdot x=1_Gx=x \quad \forall x\in X$,
\item $(gh)\cdot x = (gh)x=g(hx)=g\cdot(h\cdot x) \quad \forall g,h\in G,\ x\in X$.
\end{enumerate}
\item The action of $G$ on itself by conjugation: again let $X:=G$. Define $\cdot :G\times X\rightarrow X$ by $g\cdot x := gxg^{-1}$. Note that
\begin{enumerate}
\item $1_G\cdot x = 1_G x 1_G^{-1}=x \quad \forall x\in X$,
\item $(gh)\cdot x=(gh)x(gh)^{-1}=ghxh^{-1}g^{-1}=g\cdot \left(hxh^{-1}\right)=g\cdot (h\cdot x)$.
\end{enumerate}
\item The action of $G$ on the set of left cosets of $H\leq G$: let $X:=G/H=\{gH:g\in G\}$ and define $\cdot : G\times X\rightarrow X$ by $g\cdot kH=gkH$. To see it's indeed an action is similar to 1.
\end{enumerate} 
\end{example}

\begin{prop}
\label{prop:GhomSymX}
Let $G$ be a group acting on a set $X$. Define $\phi:G\rightarrow\Sym(X)$ by $\phi(g)(x):=g\cdot x$. Then $\phi$ is a homomorphism. (Then $G/\ker\phi\cong H$ where $H\leq \Sym(X)$).
\end{prop}
\begin{proof}
Let $g,h\in G$. $\phi$ is indeed a bijection by definition of an action. It suffices to show $\phi(gh)=\phi(g)\circ\phi(h)$. Let $x\in X$, then
\[
\phi(gh)(x)=(gh)\cdot x = g\cdot(h\cdot x)=\phi(g)(\phi(h)(x))=(\phi(g)\circ\phi(h))x.
\]
\end{proof}

\begin{defn}
Let $\phi$ be the same map as above.
\begin{enumerate}
\item The \textit{kernel of action} of $G$ on $X$, denoted $\ker(G,X,\cdot)$, is defined to be
\[
\ker(G,X,\cdot)=\ker\phi=\{g\in G:g\cdot x=x \ \forall x\in X\} \unlhd G.
\]
\item The \textit{image} of the action, denoted $\im(G,X,\cdot)$, is defined to be $\im\phi\leq\Sym(X)$.
\item The action is \textit{trivial} if $\ker(G,X,\cdot)=G$ and \textit{faithful} if $\ker(G,X,\cdot)=\{1_G\}$.
\end{enumerate}
\end{defn}

\begin{example}[The same ones from \ref{example:gpaction}]
\begin{enumerate}
\item $\ker(G,X,\cdot)=\{1_G\}$, a faithful action.
\item $\ker(G,X,\cdot)=\{g\in G:gxg^{-1}=x \ \forall x\in X\}=Z(G)$. The action is trivial iff $G$ is abelian.
\item Observe that the action is trivial $\Leftrightarrow gxH=xH \ \forall g,x\in G\Leftrightarrow H=G$, i.e. it's nontrivial as long as $H$ is proper. This is useful: let $G$ be a nonabelian finite simple group. We claim $G$ cannot have a subgroup of index 3 (the case that index is 2 is obvious since if that's true then it has a nontrivial proper normal subgroup, so not simple).
\begin{proof}
Suppose $|G:H|=3.\ G$ acts on $X:=G/H$ and by the above $H$ is proper, so $K:=\ker(G,X,\cdot) \unlhd G$ is proper. But $G$ is simple so $K=\{1_G\}$ and one can then say $G\cong G/K\cong$ some subgroup of $S_3$. Since it's nonabelian it must be the whole group. But $S_3$ is not simple, a contradiction. 
\end{proof}
\end{enumerate}
\end{example}

\begin{flushright}
\textit{Week 4, lecture 1 starts here}
\end{flushright}

\begin{remark}
We saw last time that Proposition \ref{prop:GhomSymX} is particularly useful when $G$ is a finite simple group and $H$ is a subgroup of $G$ such that $|G:H|=n$, in that it implies that $G$ is isomorphic to a subgroup of $S_n$. This leads to the following more general result.
\end{remark}

\begin{prop}
\label{prop:faithfulisosgpsymX}
Let $G$ be a group acting faithfully on a set $X$. Then $G$ is isomorphic to a subgroup of $\Sym(X)$.
\end{prop}
\begin{proof}
This follows immediately from the definition of faithful and the 1st isomorphism theorem.
\end{proof}

\begin{defn}
Let $G$ be a group acting on a set $X$ and $x\in X$.
\begin{enumerate}
\item The \textit{orbit} of $x$ is $\orb_G(x):=\{g\cdot x:g\in G\}$.
\item The \textit{stabiliser} of $x$ is $\stab_G(x):=\{g\in G:g\cdot x=x\}$.
\end{enumerate}
\end{defn}

\begin{prop}
\begin{enumerate}
\item $\stab_G(x)\leq G$.
\item $\displaystyle \ker (G,X,\cdot)=\bigcap_{x\in X} \stab_G(x)$.
\end{enumerate}
\end{prop}
\begin{proof}
See sheet 2 Q8.
\end{proof}

\begin{example}[From \ref{example:gpaction}.2]
Fix $x\in X=G$. One has
\[
\orb_G(x)=\{gxg^{-1}:g\in G\},
\]
called the \textit{conjugacy class} of $x$ in $G$, sometimes denoted $^G x$. Also
\[
\stab_G(x)=\{g\in G:gxg^{-1}=x\},
\]
called the \textit{centraliser} of $x$ in $G$, sometimes denoted $C_G(x)$.
\end{example}

\begin{thm}[Orbit-stabiliser]
Let $G$ be a finite group acting on a set $X$ and $x\in X$. Then
\[
|G:\stab_G(x)|=|\orb_G(x)|,
\]
or alternatively
\[
|G|=|\stab_G(x)||\orb_G(x)|.
\]
\end{thm}
\begin{proof}
Let $S=\stab_G(x)$. Recall $G/S=\{gS:g\in G\}$ and $|G:S|=|G/S|$. Define
\[
f:G/S\rightarrow\orb_G(x) \text{ by } gS\mapsto g\cdot x.
\]
It suffices to show $f$ is bijective.
\begin{enumerate}
\item $f$ is well-defined and injective: $gS=kS\Leftrightarrow k^{-1}g\in S \Leftrightarrow k^{-1}g\cdot x=x \Leftrightarrow g\cdot x=k\cdot x\Leftrightarrow f(gS)=f(kS)$;
\item For $g\cdot x\in\orb_G(x)$ then $f(gS)=g\cdot x$, so $f$ is surjective.
\end{enumerate}
\end{proof}

\begin{coro}
\label{coro:orbpartitionX}
\begin{enumerate}
\item For $x,y\in X$, either $\orb_G(x)=\orb_G(y)$ or $\orb_G(x)\cap \orb_G(y)=\varnothing$.
\item $\{\orb_G(x):x\in X\}$ is a partition of $X$.
\item $|\orb_G(x)|$ divides $|G|$.
\end{enumerate}
\end{coro}
\begin{proof}
\begin{enumerate}
\item[1, 2.] Define a relation on $X$ $x\sim y$ if $y=g\cdot x$. It follows from the definition of an action that $\sim$ is an equivalence relation and the equivalence classes are $\{\orb_G(x):x\in X\}$.
\item[3.] Immediate from the theorem.
\end{enumerate}
\end{proof}

\begin{thm}[Cayley's]
Let $G$ be a finite group. Then $G$ is isomorphic to a subgroup of $\Sym(X)$ for some set $X$.
\end{thm}
\begin{proof}
By Example \ref{example:gpaction}.1, $G$ acts on itself by left multiplication, and $\ker(G,X,\cdot)=\{1_G\}$, i.e. the action is faithful. The result then follows from Proposition \ref{prop:faithfulisosgpsymX}.
\end{proof}

\begin{thm}
\label{thm:primepowerordergpnontrivZ}
Let $p$ be prime and $G$ a group of order $p^n$ where $n\in\mathbb N^+$. Then $|Z(G)|>1$.
\end{thm}
\begin{proof}
Observe that
\[
g\in Z(G)\Leftrightarrow gxg^{-1}=x \ \forall x\in G \Leftrightarrow xgx^{-1}=g \Leftrightarrow |\orb_G(g)|=1.
\]

\begin{flushright}
\textit{Week 4, lecture 2 starts here}
\end{flushright}

Let $\orb_G(x_1),\ldots,\orb_G(x_t)$ be the orbits of $G$ in its action by conjugation on $X=G$ (Example \ref{example:gpaction}.2). Assume WLOG that $|\orb_G(x_i)|=1$ for $1\leq i\leq s$ and $|\orb_G(x_i)|>1$ for $s<i\leq t$. By the observation above, one then has $Z(G)=\{x_1,\ldots,x_s\}$ and in particular, $|Z(G)|=s$. If $s<i\leq t$, then $|\orb_G(x)|=p^{a_i}$ for some $a_i\in \mathbb N$ by Corollary \ref{coro:orbpartitionX}.3. Now, by Corollary \ref{coro:orbpartitionX}.2,
\[
|G|=|X|=\sum_{i=1}^t \left|\orb_G(x_i)\right|=s+\sum_{i=s+1}^t p^{a_i}=p^n,
\]
so $|Z(G)|=s\equiv 0\Mod p$, hence $|Z(G)|\neq 1$.
\end{proof}

\begin{remark}
Many groups we shall see in the course will have a trivial centre, e.g. $S_n$ for $n\geq 3$ and $D_{2n}$ for $n\geq 3$. Also, a nonabelian finite simple group is not of order $p^n$.
\end{remark}

\begin{coro}
Let $p$ be prime and $G$ a group.
\begin{enumerate}
\item $|G|=p^2\Rightarrow G$ is abelian.
\item $|G|=p^3\Rightarrow$ either $G$ is abelian or $|Z(G)|=p$.
\end{enumerate}
\end{coro}
\begin{proof}
We need two facts:
\begin{enumerate}
\item All groups of order $p$ are cyclic (immediate from Lagrange).
\item If $G$ is nonabelian then $G/Z(G)$ is not cyclic (see sheet 2 Q1).
\end{enumerate}
It follows that if $G$ is nonabelian then $|G/Z(G)|\neq p$ for a prime $p$. Now Theorem \ref{thm:primepowerordergpnontrivZ} implies
\begin{enumerate}
\item $|G|=p^2\Rightarrow |Z(G)|=p^2\Rightarrow Z(G)=G \Rightarrow G$ is abelian.
\item $|G|=p^3\Rightarrow |Z(G)|=p$ or $p^3$ and the desired result is clear.
\end{enumerate}
\end{proof}

\begin{thm}[Cauchy's]
Let $G$ be a finite group and $p$ a prime divisor of $|G|$. Then $G$ has an element of order $p$. Furthermore, number of elements of order $p$ is congruent to $-1\Mod p$.
\end{thm}
\begin{proof}
Define
\[
X:=\{(g_1,\ldots,g_p)\in G^p:g_1\cdots g_r=1_G\}.
\]
Note that
\[
\begin{aligned}
x=(g_1,\ldots,g_p)\in X &\Rightarrow 1_G=g_1\cdots g_p \\
&\Rightarrow g_i^{-1}\cdots g_1^{-1} 1_G g_1\cdots g_i=g_i^{-1}\cdots g_1^{-1} g_1\cdots g_p g_1\cdots g_i \\
&\Rightarrow 1_G=g_{i+1}\cdots g_pg_1\cdots g_i \\
&\Rightarrow (g_{i+1},\ldots ,g_p,g_1,\ldots ,g_i)\in X.
\end{aligned}
\]
Now define
\[
C:=\langle\sigma\rangle\leq S_p \text{ where }\sigma=(1,2,\ldots,p)
\]
and the action
\[
\cdot :C\times X\rightarrow X \text{ by } \sigma^i\cdot (g_1,\ldots,g_p):=(g_{i+1},\ldots ,g_p,g_1,\ldots ,g_i).
\]
(Check $\cdot$ is indeed an action.) Now
\begin{enumerate}
\item If $g\in G$ and $g^p=1_G$ then $(g,\ldots,g)\in X$, and $\sigma^i\cdot(g,\ldots,g)=(g,\ldots,g) \ \forall i$, i.e. $|\orb_C((g,\ldots,g))|=1$.
\item We claim that the converse is true: if $x$ satisfies $|\orb_C(x)|=1$ then $x=(g,\ldots,g)$ for some $g\in G : g^p=1_G$. Indeed, say $x=(g_1,\ldots,g_p)$. It suffices to show $g_1=g_i \ \forall i$. By the orbit-stabiliser theorem, $|\orb_C(x)|=1$ implies $\stab_C(x)=C$, i.e. $\forall i,$
\[
(g_1,\ldots,g_p)=\sigma^{i-1}(g_1,\ldots,g_p)=(g_i,\ldots ,g_p,g_1,\ldots ,g_{i-1}),
\]
which gives the desired.
\item Note that if $(g_1,\ldots,g_p)\in X$ then $g_p=(g_1\cdots g_{p-1})^{-1}$. We claim $|X|=|G|^{p-1}$. Indeed, define $f:X\rightarrow G^{p-1}$ by $(g_1,\ldots,g_p)\mapsto (g_1,\ldots,g_{p-1})$. It suffices to show that $f$ is bijective since then $|X|=|G^{p-1}|=|G|^{p-1}$. To see $f$ is injective, note that
\[
\begin{aligned}
f((g_1,\ldots,g_p))=f((h_1,\ldots,h_p)) &\Rightarrow g_i=h_i \text{ for } 1\leq i\leq p-1 \\
&\Rightarrow g_p=(g_1\cdots g_{p-1})^{-1}=(h_1\cdots h_{p-1})^{-1}=h_p \\
&\Rightarrow (g_1,\ldots,g_p)=(h_1,\ldots,h_p).
\end{aligned}
\]
To see $f$ is surjective, note that for every $(x_1,\ldots,x_{p-1})\in G^{p-1}$ one can set $x_p:=(x_1\cdots x_{p-1})^{-1}$, then $(x_1,\ldots,x_p)\in X$ and it satisfies $f((x_1,\ldots,x_p))=(x_1,\ldots,x_{p-1})$.
\end{enumerate}
By Corollary \ref{coro:orbpartitionX}.3, all orbits not of size 1 have size $p$. Let $s$ be number of distinct orbits of size 1, $t$ be number of distinct orbits of size $p$ and $r$ be number of elements of order $p$ in $G$. By parts 1 and 2, $s=1+r$ where 1 corresponds to the trivial element $(1_G,\ldots,1_G)$. One can then write $|G|^{p-1}=|X|=1+r+pt$, and since $p\mid |G|,\ r\equiv -1\Mod p$. In particular, $r>0$.
\end{proof}

\begin{flushright}
\textit{Week 4, lecture 3 starts here}
\end{flushright}

\textbf{Recap}: We now have three nice tools for analysing element orders in a finite group $G$. Let $E_p(G):=\{x\in G:|x|=p\}$ where $p$ prime. Then
\begin{enumerate}
\item $|E_p(G)|\equiv -1\Mod p$ (Cauchy's theorem)
\item $|E_p(G)|\leq |G:C_G(x)| \ \forall x\in G$ by \ref{exe:conjaugatesordersequal}.1 and the orbit-stabiliser theorem.
\item If $r\neq p$ is a prime and $G$ has no element of order $pr$, then $|C_G(x)|$ is not divisible by $r$ for $x\in E_p(G)$ by Lemma \ref{lemma:orderlem}.4 and Cauchy's theorem.
\end{enumerate}

\begin{example}
Let $G$ be of order 48 with no elements of order 6. We claim $|E_3(G)|\geq 17$.
\begin{proof}
Let $x\in E_3(G)$. Tool 3 implies $|C_G(x)|$ is not divisible by 2. Since $|C_G(x)|\mid 48$, it must be $|C_G(x)|=3$. Then by tool 2 $|E_3(G)|\geq 16$, and since $|E_3(G)|\equiv -1\Mod 3,\ |E_3(G)|\geq 17$.
\end{proof}
\end{example}

\begin{prop}
\label{prop:orderofHK}
Let $G,H,X$ be as in Example \ref{example:gpaction}.3 and $K\leq G$. Then $|KH|=\frac{|K||H|}{|K\cap H|}$.
\end{prop}
\begin{proof}
Since $G$ acts on $X$ and $K\leq G, K$ acts on $X$ as well. Let $x=H\in X$. Then
\[
\stab_K(x)=\{k\in K:kH=H\}=\{k\in K:k\in H\}=K\cap H,
\]
and
\[
|K:K\cap H|=|\orb_K(x)|=|\{kH:k\in K\}|.
\]
On the other hand,
\[
|KH|=\left|\bigcap_{k\in K} kH\right| = |\{kH:k\in K\}||H|=|K:K\cap H||H|.
\]
\end{proof}

\begin{coro}
Let $G,H,K$ as above. Then
\[
|G:H\cap K|\leq |G:H||G:K|.
\]
\end{coro}
\begin{proof}
\[
\frac{|H||K|}{|H\cap K|}=|KH|\leq |G|=\frac{|G|^2}{|G|},
\]
and rearranging gives the desired.
\end{proof}

\subsection{Fixed points}
\begin{defn}
Let $G$ be a group acting on a set $X$ and $g\in G$.
\begin{enumerate}
\item An element $x\in X$ is a \textit{fixed point} of $g$ if $g\cdot x=x$. The set of fixed points of $g$ is denoted $\fix_X(g):=\{x\in X:g\cdot x=x\}$.
\item $g$ is \textit{fixed point free} if $\fix_X(g)=\varnothing$.
\end{enumerate}
\end{defn}

\begin{lemma}[not Burnside's\footnote{William Burnside (1852–1927) was known as a pioneer in the systematic study of finite groups and indeed stated and proved this lemma, but later people found out this equality was known in as early as 1845 to Cauchy, so it's \textit{a lemma that is not Burnside's}.}]
Let $G$ be a finite group acting a finite set $X$. Then
\[
|\{\orb_G(x):x\in X\}|=:r=\frac{1}{|G|}\sum_{g\in G}|\fix_X(g)|.
\]
Informally, the number of orbits = the average number of fixed points.
\end{lemma}
\begin{proof}
We will use Corollary \ref{coro:orbpartitionX}.1 and 2. Let
\[
\Lambda = \{(g,x):g\in G,x\in X, g\cdot x=x\}.
\]
We count $|\Lambda|$ in two different ways (double-counting method to show equality).
\begin{enumerate}
\item \[
|\Lambda|=\sum_{g\in G} |\fix_X(g)|.
\]
\item \[
\begin{aligned}
|\Lambda|&=\sum_{x\in X} |\{g\in G:g\cdot x=x\}|=\sum_{x\in X} |\stab_G(x)|=\sum_{x\in X} \frac{|G|}{|\orb_G(x)|}\\
&=\sum_{i=1}^r\sum_{y\in\orb_G(x_i)} \frac{|G|}{|\orb_G(y)|}=\sum_{i=1}^r\sum_{y\in\orb_G(x_i)} \frac{|G|}{|\orb_G(x_i)|} \\
&=\sum_{i=1}^r |\orb_G(x_i)| \frac{|G|}{|\orb_G(x_i)|}=r|G|
\end{aligned}
\]
where $\orb_G(x_1),\ldots,\orb_G(x_r)$ are distinct orbits.
\end{enumerate}
\end{proof}

\begin{coro}
Let $G,X$ and $r$ be as in above lemma. Suppose $|X|>1$ and $r=1$. Then $G$ has a fixed point free element.
\end{coro}
\begin{proof}
By definition one has $|\fix_X(1_G)|=|X|$. Now
\[
1=\frac{1}{|G|}\sum_{g\in G} |\fix_X(g)|=\frac{1}{|G|}\left(|\fix_X(1_G)|+\sum_{g\neq 1_G} |\fix_X(g)|\right).
\]
So if $G$ doesn't have any fixed point free element then $|\fix_X(g)|\geq 1 \ \forall g\in G$ and
\[
1\geq \frac{1}{|G|}(|X|+|G|-1)>\frac{|G|}{|G|}=1,
\]
a contradiction.
\end{proof}

\begin{flushright}
\textit{Week 5, lecture 1 starts here}
\end{flushright}

\section{Sylow theorems}
\begin{remark}[Philosophy]
In chapter 1, we saw Lagrange's theorem. Question: does the converse hold? i.e., if $l\mid |G|$, does $G$ necessarily have a subgroup of order $l$?
\begin{enumerate}
\item A counterexample would be $A_4$ with $|A_4|=12$, which does not have a subgroup of order 6 (use tool 3).
\item In general, let $G$ be a finite simple group of even order $>2$. Then $G$ has no subgroup of order $|G|/2$.
\end{enumerate}
Sylow theorems will prove that a partial converse holds by restricting $l$.
\end{remark}

\begin{notation}
For the remainder of the chapter, we fix a finite group $G$ and a prime divisor $p$ of $|G|$. Also, we write $|G|_p$ for the $p$-part of $|G|$, i.e. writing $|G|=p^n m$ where $p\nmid m$ we have $|G|_p=p^n$.
\end{notation}

\begin{defn}
Let $H\leq G$.
\begin{enumerate}
\item $H$ is a $p$\textit{-subgroup} of $G$ if $|H|$ is a power of $p.$
\item $H$ is a \textit{Sylow} $p$\textit{-subgroup} of $G$ if $|H|=|G|_p$.
\item The set of all Sylow $p$-subgroups of $G$ is denoted $\Syl_p(G)$.
\end{enumerate}
\end{defn}

\begin{example}
\begin{enumerate}
\item $G=S_4$ has order 24. Then $|G|_2=2^3,\ |G|_3=3$. One has $\langle (1,2,3)\rangle\in\Syl_3(G)$ and $D_8=\langle (1,2,3,4),(1,4)(2,3)\rangle\in\Syl_2(G)$. Also $\langle(1,2)\rangle$ is a 2-subgroup but not a Sylow 2-subgroup.
\item $G=C_n$. Then for each divisor $d$ or $n,\ G$ has a unique subgroup of order $d$. In particular, if $p\mid n$, then $|\Syl_p(G)|=1$. See sheet 2 Q3.
\item $G=GL_2(F)$ where $F$ is a field of order $p$. Then by Theorem \ref{thm:orderofGLnF}, $|G|=p^{\binom{2}{2}} \prod_{i=1}^2(p^i-1)=p(p-1)(p^2-1)$. One has $x=\begin{pmatrix}1&1\\0&1\end{pmatrix}\in G$ with order $p$. Hence $\langle x\rangle \in \Syl_p(G)$. More generally, $|GL_n(F)|_p=p^{\binom{n}{2}}$ and $U(n,F)$ (the set of upper triangular matrices with 1 on the diagonal) is a Sylow $p$-subgroup.
\end{enumerate}
\end{example}

\begin{thm}[Sylow theorems]
Let $G$ be a finite group with $p$ a prime divisor of $|G|$.
\begin{enumerate}
\item (Existence) $\Syl_p(G)\neq\varnothing$.
\item (Conjugacy) All Sylow $p$-subgroups are conjugate in $G$.
\item (Containment) Every $p$-subgroup of $G$ is contained in a Sylow $p$-subgroup.
\item (Number) $|\Syl_p(G)|\equiv 1\Mod p$.
\end{enumerate}
\end{thm}

\subsection{Wielandt's proof of Sylow theorems 1 \& 4}
\begin{lemma}
\label{lemma:1stlemSylow4}
Let $p$ be prime and $n,m\in\mathbb N^+$ with $\gcd(m,p)=1$. Then
\begin{enumerate}
\item $\displaystyle p\mid \binom{p}{i}$ for $1\leq i\leq p-1$.
\item $\displaystyle \binom{p^n m}{p^n}\equiv m\Mod p$.
\end{enumerate}
\end{lemma}
\begin{proof}
\begin{enumerate}
\item Fix $1\leq i\leq p-1$. Then
\[
\binom{p}{i}=\frac{p!}{i!(p-i)!}=\frac{p(p-1)\cdots(p-i+1)}{i(i-1)\cdots 1}.
\]
Now let $a:=(p-1)\cdots (p-i+1),\ b=i!$. Then
\[
\binom{p}{i}=\frac{pa}{b}\Rightarrow pa=b\binom{p}{i}\Rightarrow p\mid b\binom{p}{i},
\]
but clearly $\gcd(p,b)=1$, hence $\displaystyle p\mid \binom{p}{i}$.
\item Let $F:=\mathbb Z/p\mathbb Z=\{0,1,\ldots,p-1\}$ with usual addition and multiplication modulo $p$. Consider the polynomial $(1+x)^p\in F[x]$.
\begin{flushright}
\textit{Week 5, lecture 2 starts here}
\end{flushright}
By binomial theorem,
\[
(1+x)^p=\sum_{i=1}^p \binom{p}{i}x^i=1+x^p \in F[x].
\]
Then
\[
(1+x)^{p^2}=\left((1+x)^p\right)^p=\left(1+x^p\right)^p=1+x^{p^2}.
\]
Inductively,
\[
(1+x)^{p^n}=1+x^{p^n}.
\]
Even more generally,
\[
(1+x)^{p^n m}=\left((1+x)^{p^n}\right)^m=\left(1+x^{p^n}\right)^m.
\]
Binomial theorem then gives us the equality
\[
\sum_{i=0}^{p^n m} \binom{p^n m}{i} x^i = \sum_{i=0}^{m} \binom{m}{i} x^{p^n i}.
\]
Comparing coefficients of $x^{p^n i}$ gives
\[
\binom{p^n m}{p^n i}=\binom{m}{i}
\]
and in particular for $i=1$,
\[
\binom{p^n m}{p^n}=m \in F.
\]
Translating this back to $\mathbb Z$ one has the desired.
\end{enumerate}
\end{proof}

\begin{prop}
Sylow theorem 4. In particular, Sylow theorem 1.
\end{prop}
\begin{proof}
As usual, write $|G|=p^n m$ where $p\nmid m$ and $p^n=:|G|_p$. Let $X:=\{S\subseteq G:|S|=|G|_p\}$. Define $\cdot G\times X\rightarrow X$ by $g\cdot S:=gS=\{gs:s\in S\}$. This is indeed an action: see sheet 2 Q12. Let $\orb_G(S_i)$ be $t$ distinct orbits in $X$. By Corollary \ref{coro:orbpartitionX}.2 and Lemma \ref{lemma:1stlemSylow4}.2,
\[
\binom{p^n m}{p^n}=|X|=\sum_{i=1}^t |\orb_G(s_i)| \equiv m\Mod p.
\]
This means at least one $|\orb_G(s_i)|$ is not divisible by $p$. WLOG, suppose $p\nmid |\orb_G(S_i)|$ for $1\leq i\leq r$ and $p\mid |\orb_G(S_i)|$ for $r< i\leq t$. We claim:
\begin{enumerate}
\item Fix $i=1,\ldots,r$ and denote $S_i$ by $S$ for convenience. Then $\exists x\in G:\stab_G(xS)=xS$ and in particular $xS\in\Syl_p(G)$. Indeed, let $s\in S$ and set $x=s^{-1},\ T:=xS$. We want to show $\stab_G(T)=T$. First note that $1_G=xx^{-1}=xs\in T$. Hence $g\in\stab_G(T)\Rightarrow gT\Rightarrow g=g1_G\in gT=T$, so $\stab_G(T)\subseteq T$. Also, $T\in\orb_G(S)$, so $\orb_G(T)=\orb_G(S)$. Hence
\[
p\nmid |\orb_G(T)|=\frac{|G|}{|\stab_G(T)|}=\frac{p^n m}{|\stab_G(T)|}.
\]
This implies $p^n\mid |\stab_G(T)|$ by Lagrange's theorem. But by construction, $|T|=p^n$, so it must be that $\stab_G(T)=T$.
\item $r=|\Syl_p(G)|$. Indeed, for $i=1,\ldots,r$ we can take $T_i=x_i S_i\in\orb_G(S_i)$ such that $T_i=\stab_G(T_i)$ by previous claim. Now define
\[
\begin{aligned}
f:\{\orb_G(T_1),\ldots,\orb_G(T_r)\} &\rightarrow \Syl_p(G) \\
\orb_G(T_i) &\mapsto T_i
\end{aligned}
\]
$f$ is well-defined since $\orb_G(T_i)$ are distinct by construction and $T_i\in\Syl_p(G)$ by first claim. Since $T_i$ are distinct, $f$ is injective. Now let $P\in\Syl_p(G)$. Then $P\in X$, and
\[
\stab_G(P)=\{g\in G:gP=P\}=P,
\]
so $|\orb_G(P)|=m$ which by definition is not divisible by $p$. Hence for some $i=1,\ldots,r,\ \orb_G(P)=\orb_G(T_i)$, so $P\in\orb_G(T_i)$, i.e. $P=gT_i$ for some $g\in G$. But $g=g1_G\in gT_i=P$ and since $g^{-1}\in P$, $T_i=g^{-1}P=P$. This proves $f$ is surjective, hence bijective, hence the claim.
\end{enumerate}
Therefore,
\[
rm+0=\sum_{i=1}^r |\orb_G(T_i)|+\sum_{i=r+1}^t |\orb_G(S_i)|=|X|\equiv m\Mod p
\]
and since $\gcd(m,p)=1,$ we can do cancellation and have $r\equiv 1\Mod p$.
\end{proof}

\begin{flushright}
\textit{Week 5, lecture 3 starts here}
\end{flushright}

\subsection{Proofs of Sylow theorems 2 \& 3}
\begin{remark}[Easy but useful facts]
Let $G$ be finite and $p$ a prime divisor of $|G|$. Then
\begin{enumerate}
\item $p\in\Syl_p(G),g\in G\Rightarrow gPg^{-1}\in\Syl_p(G)$.
\item If $|G|$ is a power of $p$ then $\Syl_p(G)=\{G\}$.
\item By definition, a $p$-subgroup $Q$ of $G$ is a Sylow $p$-subgroup iff $p\nmid |G:Q|$.
\end{enumerate}
\end{remark}

\begin{prop}
Let $G,p$ be as above and $P\in\Syl_p(G),\ H\leq G$. Then $\exists g\in G:H\cap gPg^{-1}\in\Syl_p(H)$.
\end{prop}
\begin{proof}
Let $X=G/P=\{gP:g\in G\}$. Then $H$ acts on $X$ by left multiplication (since $G$ does) (Example \ref{example:gpaction}.3). Consider the orbits and stabilisers. Fix $xP\in X$ where $x\in G$, then
\[
\begin{aligned}
\stab_H(xP)&=\{h\in H:hxP=xP\}=\{h\in H:x^{-1}hxP=P\}\\
&=\{h\in H:x^{-1}hx\in P\}=\{h\in H:h\in xPx^{-1}\}=H\cap xPx^{-1}.
\end{aligned}
\]
As usual, let $\orb_H(x_1P),\ldots,\orb_H(x_tP)$ be distinct orbits and write $|G|=p^n m$ where $p\nmid m$. We have
\[
p\nmid m=|X|=\sum_{i=1}^t \left|\orb_H(x_i P)\right|=\sum_{i=1}^t \left|H:\left(H\cap x_i Px_i^{-1}\right)\right|
\]
so $p\nmid \left|H:\left(H\cap x_i Px_i^{-1}\right)\right|$ for some $i$. We claim $g:=x_i$ satisfies the desired. Indeed, $H\cap gPg^{-1} \leq gPg^{-1}$, so by Lagrange's theorem it's a $p$-subgroup of $H$, hence by 3rd remark above it's a Sylow $p$-subgroup of $H$.
\end{proof}

\begin{coro}
Sylow theorems 2 and 3.
\end{coro}
\begin{proof}
\begin{enumerate}
\item[2.] Let $H,P\in\Syl_p(G)$. Then $\exists g\in G:H\cap gPg^{-1}\in \Syl_p(H)=\{H\}$ by previous proposition and the 2nd remark above. So $H=H\cap gPg^{-1}$, in particular $H\subseteq gPg^{-1}$, but by assumption $|H|=|gPg^{-1}|$ so $H=gPg^{-1}$.
\item[3.] Let $H\leq G$ be a $p$-subgroup and $P\in\Syl_p(G)$. Then by exactly the same argument as above, $H\subseteq gPg^{-1}\in\Syl_p(G)$.
\end{enumerate}
\end{proof}

\subsection{Consequences of Sylow theorems}
Recall that if $H\leq G$ then $H\leq N_G(H)=\{g\in G:gHg^{-1}=H\}$.

\begin{coro}
\label{coro:consequence1Sylow}
Let $G,p$ be as above and $P\in \Syl_p(G)$.
\begin{enumerate}
\item $|\Syl_p(G)|=|G:N_G(P)|$.
\item $\Syl_p(G)\mid |G:P|$.
\item $P\unlhd G\Leftrightarrow |\Syl_p(G)|=1$.
\end{enumerate}
\end{coro}
\begin{proof}
Let $G$ acts on $X:=\Syl_p(G)$ by conjugation (see sheet 2 Q15 that this is indeed an action).
\begin{enumerate}
\item By Sylow theorem 2, $\Syl_p(G)$ is explicitly $\{gPg^{-1}:g\in G\}$ which by definition is $\orb_G(P)$. Now $\stab_G(P)=\{g\in G:gPg^{-1}=P\}=N_G(P)$. The desired result then follows from orbit-stabiliser theorem.
\item By Lagrange's theorem and part 1, $P\leq N_G(P)\Rightarrow |P| \mid |N_G(P)| \Rightarrow |G:N_G(P)| \mid |G:P|\Rightarrow |\Syl_p(G)| \mid |G:P|$.
\item We have $P\unlhd G\Leftrightarrow \{gPg^{-1}:g\in G\}=\{P\}\Leftrightarrow \Syl_p(G)=\{P\}\Leftrightarrow |\Syl_p(G)|=1$.
\end{enumerate}
\end{proof}

\begin{coro}
\label{coro:consequence2Sylow}
Let $G,p$ be as above and
\[
F_p(G):=\{x\in G:x\neq 1_G,\ |x|=p^n\}.
\]
Then
\begin{enumerate}
\item \[
F_p(G)=\bigcup_{P\in\Syl_p(G)} P\backslash \{1_G\}
\]
\item $|F_p(G)|\geq |G|_p-1$ with equality iff $|\Syl_p(G)|=1$ (i.e. there is a normal Sylow $p$-subgroup).
\item If $|G|_p=p$, then $|F_p(G)|=|\Syl_p(G)|(p-1)$.
\end{enumerate}
\end{coro}

\begin{flushright}
\textit{Week 6, lecture 1 starts here}
\end{flushright}

\begin{proof}
\begin{enumerate}
\item Let \[
x\in \bigcup_{P\in\Syl_p(G)} P\backslash \{1_G\}.
\]
Then $|x|=p^n$ by Lagrange's, and since $x\neq 1$ one has $x\in F_p(G)$. We haven't used Sylow yet. Now let $x\in F_p(G)$. Then $\langle x\rangle$ is a $p$-subgroup since its order is $|x|$, so $\langle x\rangle$ is contained in a Sylow $p$-subgroup. The desired is then clear
\item[2, 3.] See sheet 3 Q10, 11 respectively.
\end{enumerate}
\end{proof}

\begin{example}[Applying \ref{coro:consequence1Sylow} and \ref{coro:consequence2Sylow}]
\label{example:strategy2}
\begin{enumerate}
\item Prove that a group of order 30 is not simple.
\begin{proof}
Suppose $|G|=30$ and $G$ is simple. Note $|G|=2\times 3\times 5$. By Corollary \ref{coro:consequence1Sylow}.2 and Sylow theorem 4, $|\Syl_5(G)|\mid 6$ and $|\Syl_5(G)|\equiv 1\Mod 5$, i.e. $|\Syl_5(G)|=1$ or 6. If it's 1 then by Corollary \ref{coro:consequence1Sylow}.3 $G$ is not simple with $P$ normal, a contradiction; so $|\Syl_5(G)|=6$. Similarly, $|\Syl_3(G)|=10$. Now Corollary \ref{coro:consequence2Sylow}.3 says $|F_5(G)|=6\times 4=24$ and $|F_3(G)|=10\times 2=20$, but we only have 30 elements. Hence $G$ must be not simple.
\end{proof}
\item Prove that a group of order 132 is not simple.
\begin{proof}
Suppose $|G|=132=11\times 2^2\times 3$ and $G$ is simple. Then similarly, $|\Syl_{11}(G)| \mid 12$ and $|\Syl_{11}(G)|\equiv 1\Mod 11$, i.e. $|\Syl_{11}(G)|=1$ or 12. But again $G$ has no normal subgroup, so $|\Syl_{11}(G)|=12$. Similarly, $|\Syl_3(G)|=4$ or 22. Again, $|F_{11}(G)|=12\times 10=120$ and $|F_3(G)|\geq 4\times 2=8$. Now,
\[
F_2(G)\subseteq G\backslash F_{11}(G) \sqcup F_3(G) \sqcup \{1_G\},
\]
so
\[
|F_2(G)|\leq 132-120-8-1=3.
\]
Corollary \ref{coro:consequence2Sylow}.2 says $|F_2(G)|\geq 2^2-1=3$, so $|F_2(G)|=3$, hence there is a normal Sylow $p$-subgroup, a contradiction with $G$ being simple.
\end{proof}
\end{enumerate}
\end{example}

\subsection{2 applications of Sylow theorems}
In this section, we'll look at a game with 2 versions.
\begin{itemize}
\item Version 1: Prove that a group $G$ of order $\ast$ is not simple. The 3 strategies are
\begin{enumerate}
\item Immediately apply Corollary \ref{coro:consequence1Sylow}.2 and Sylow theorem 4 to try to get a contradiction. We usually start with the largest $p$.

\textbf{e.g.} $\ast=20=2^2\times 5$. Then $|\Syl_5(G)|=1$, an immediate contradiction.
\item The $F_p(G)$-strategy: for each $p$ such that $|G|_p=p$, use Corollary \ref{coro:consequence2Sylow}.3 to get a lower bound on $|F_p(G)|$. Since 
\[
|G|<\sum_{p\mid |G|}|F_p(G)|,
\]
we either get an immediate contradiction or we should further use Corollary \ref{coro:consequence2Sylow}.3 to get one.

\textbf{e.g.} Example \ref{example:strategy2}.

\begin{flushright}
\textit{Week 6, lecture 2 starts here}
\end{flushright}

\item The homomorphism strategy: again begin by considering possibilities for $|\Syl_p(G)|$. Note that if we choose a $p$ such that $|G:N_G(P)|=|\Syl_p(G)|=m>1$ for $P\in\Syl_p(G)$ (Corollary \ref{coro:consequence1Sylow}), then $\ker(G,\Syl_p(G),\cdot)\subseteq \stab_G(P)=N_G(P) \subsetneqq G$ is proper. Since we assume (for contradiction) that $G$ is simple, $\ker(G,\Syl_p(G),\cdot)=\{1_G\}$ because otherwise it would be a nontrivial, proper normal subgroup. Hence by Proposition \ref{prop:faithfulisosgpsymX}, $G\cong$ some subgroup of $\Sym(X)$ and in particular $|G|\mid m!$. We would then get a contradiction hopefully.

\textbf{e.g.} $\ast=48=2^4\times 3$. Then $|\Syl_2(G)|=3$. So $G\cong$ a subgroup of $(\Sym(\Syl_2(G))\cong S_3)$ and in particular $48\mid 6$, which is absurd.
\end{enumerate}

\item Version 2: Prove that a finite group $G$ with given properties (usually conjugacy classes of elements of prime order) is simple. Essentially, use the following corollary.
\end{itemize}

\begin{coro}
\label{pactionhappeninN}
Let $N\unlhd G$ a finite group and $p$ a prime divisor of $|G|$. Then
\begin{enumerate}
\item $x\in N\Rightarrow \{gxg^{-1}:g\in G\}\subseteq N$.
\item $p\nmid |G:N| \Rightarrow \Syl_p(N)=\Syl_p(G)$ and $F_p(N)=F_p(G)$.
\end{enumerate}
\end{coro}
\begin{proof}
\begin{enumerate}
\item Immediate from definition.
\item By the 2nd isomorphism theorem, for a $P\in\Syl_p(G),\ P/(P\cap N)\cong PN/N\leq G/N$. So $|PN/N|\mid |P|$, hence by Lagrange's, $PN/P$ is a $p$-subgroup of $G/N$. But $p\nmid |G:N|$, so $PN/N=\{1_{G/N}\}$, i.e. $PN=N$, so $P\leq N$. So $|N|_p=|G|_p$, hence $\Syl_p(G)\subseteq \Syl_p(N)$. The other inclusion is clear.

Now $F_p(G)=\bigcup_{P\in\Syl_p(G)} P\backslash \{1_G\}=\bigcup_{P\in\Syl_p(N)} P\backslash \{1_G\}=F_p(N)$.
\end{enumerate}
\end{proof}

\begin{thm}
$A_5$ is simple.
\end{thm}
\begin{proof}
We need 4 facts about $G=A_5$ to start with:
\begin{enumerate}
\item $|G|=60=2^2\times 3\times 5$.
\item $G$ has 24 elements of order 5, the 5-cycles.
\item $G$ has 20 elements of order 3, the 3-cycles.
\item $G$ has 15 elements of order 2, precisely of the form $(a,b)(c,d)$ where $a,b,c,d \in \{1,\ldots,5\}$ are distinct and all such elements are conjugate.
\end{enumerate}

\begin{flushright}
\textit{Week 6, lecture 3 starts here}
\end{flushright}

Suppose $G$ is not simple and let $N\unlhd G$.
\begin{enumerate}
\item[$1^\circ$:] $p\mid |N|$ for some $p\in\{3,5\}$. Then since $|G|_p=p,\ p\nmid |G:N|$. So $F_p(G)=F_p(N)$. Hence
\begin{itemize}
\item $p=5\Rightarrow |N|\geq |F_5(N)|+1\geq 25$
\item $p=3\Rightarrow |N|\geq |F_3(N)|+1\geq 21$
\end{itemize}
so Lagrange's implies $|N|=30$, i.e. both 3 and 5 divide $|N|$. But again by Corollary \ref{pactionhappeninN}
\[
|N|\geq |F_3(N)|+|F_5(N)|+1\geq 45,
\]
a contradiction.
\item[$2^\circ$:] Neither 3 nor 5 divides $|N|$, then $|N|\mid 4$, so by Cauchy's it contains an element of order 2. Hence $N$ contains all elements of order 2, so $15\leq |N|\mid 4$, a contradiction.
\end{enumerate}
\end{proof}

\begin{lemma}
Let $X$ be the set of 3-cycles in $G=A_n$ for $n\geq 3$. Then $G=\langle X\rangle$, and if $n\geq 5$ then all 3-cycles are conjugate.
\end{lemma}
\begin{proof}
By sheet 2 Q7, every element of $A_n$ can be written as a product of an even number of transpositions. Hence it suffices to prove that $(a,b)(c,d)$ can be written as a product of 3-cycles.
\begin{enumerate}
\item[$1^\circ$:] $(a,b)=(c,d)$, then $(a,b)(c,d)=1=(1,2,3)^3$.
\item[$2^\circ$:] $|\{a,b\}\cap \{c,d\}|=1$. WLOG $a=c$. Then $(a,b)(c,d)=(a,b)(a,d)=(a,d,b).$
\item[$3^\circ$:] $\{a,b\}\cap \{c,d\}=\varnothing$, then $(a,b)(c,d)=(a,b,c)(b,c,d).$
\end{enumerate}

Now $G$ acts on $X$ by conjugation. It suffices to show $\orb_G((1,2,3))=X$. So let $(a,b,c)\in X$ with $a,b,c$ distinct. We want to find $g\in G:g(1,2,3)g^{-1}=(a,b,c)$.
\begin{enumerate}
\item[$1^\circ$:] $\{1,2,3\}\cap \{a,b,c\}=\varnothing$. Set $g=(1,2)(1,a)(2,b)(3,c)$. We add $(1,2)$ just to make $g$ even, and it doesn't effect since disjoint cycles commute and $(1,2)(a,b,c)(1,2)^{-1}=(a,b,c)(1,2)(1,2)^{-1}=(a,b,c)$.
\item[$2^\circ,\ 3^\circ$:] Similar.
\end{enumerate}
\end{proof}

\begin{lemma}
Let $n\geq 5$ and $\sigma\in A_n$. Then $\exists$ a conjugate $\sigma'\neq\sigma$ and some $i\in\{1,\ldots,n\}$ such that $\sigma(i)=\sigma'(i)$.
\end{lemma}
\begin{proof}
Let $r$ be the length of the longest cycle in $\sigma$. WLOG, we can write $\sigma=(1,2,\ldots,r)\pi$ for some $\pi\in S_n$ with $\pi$ disjoint from $(1,\ldots,r)$ and being a product of cycles of length $\leq r$.
\begin{enumerate}
\item[$1^\circ$:] $r\geq 3$. Then set $g=(3,4,5)$ and $\sigma'=g\sigma g^{-1}=g(1,\ldots,r)g^{-1}g\pi g^{-1}=(1,2,4,\ldots)g\pi g^{-1}$. So $\sigma(1)=\sigma'(1)=2$ but $\sigma(2)=3\neq 4=\sigma'(2)$.
\item[$2^\circ$:] $r\leq 2$. Left as an exercise.
\end{enumerate}
\end{proof}

\begin{remark}
\begin{enumerate}
\item Recall if $N\unlhd G$ and $H\leq G$ then $H\cap N\unlhd H$ (2nd isomorphism theorem).
\item Exercise: if $i\in\{1,\ldots,n\}$ then $\stab_{A_n}(i)\cong A_{n-1}$.
\end{enumerate}
\end{remark}

\begin{thm}
$A_n$ is simple for $n\geq 5$.
\end{thm}
\begin{proof}
Suppose $G=A_n$ is not simple and let $N\unlhd G$. We prove by induction on $n$ with base case $n=5$. By lemma above, for $1\neq \sigma\in N,\ \exists i\in\{1,\ldots,n\}$ and $g\sigma g^{-1}\neq \sigma:$
\[
\left(g\sigma g^{-1}\right)^{-1} (\sigma) (i)=i,
\]
so
\[
1\neq \left(g\sigma g^{-1}\right)^{-1} (\sigma) \in N\cap \stab_G(i) \unlhd \stab_G(i).
\]
So by induction hypothesis which says $\stab_G(i)\cong A_{n-1}$ is simple, $N\cap \stab_G(i)$ can only be the whole group $\stab_G(i)$. So $\stab_G(i)\leq N$, hence $N$ contains a 3-cycle. But then by previous lemmas, $N$ contains all 3-cycles. But $A_n$ is generated by 3-cycles, so $A_n\leq N$. Hence $A_n=N$, a contradiction.
\end{proof}

\begin{flushright}
\textit{Week 7, lecture 1 starts here}
\end{flushright}

\section{Classifying groups of small order}
\subsection{Semidirect product}
\begin{defn}
Let $H,K$ be groups. Define a binary operation $\cdot:(H\times K)\times (H\times K)\rightarrow H\times K$ by $(h_1,k_1)\cdot (h_2,k_2)=(h_1h_2,k_1k_2)$. Then $(H\times K,\cdot)$ is a group, called the \textit{direct product} of $H$ and $K$, denoted usually simply $H\times K$.
\end{defn}
\begin{remark}
\begin{enumerate}
\item One can generalise this definition to product of more than 2 groups.
\item The identity of $G_1\times \cdots \times G_t$ is $\left(1_{G_1},\ldots,1_{G_t}\right)$, and $(g_1,\ldots,g_t)^{-1}=\left(g_1^{-1},\ldots,g_t^{-1}\right)$.
\item $H\times K\cong K\times H$.
\end{enumerate}
\end{remark}

\begin{lemma}
\label{lemma:GisHKthenGcongHKifHKnormalanditsc1}
Let $H,K\unlhd G$ with $H\cap K=\{1_G\}$ and $G=HK$. Then
\begin{enumerate}
\item $hk=kh \ \forall h\in H,k\in K$.
\item $G\cong H\times K$.
\end{enumerate}
\end{lemma}
\begin{proof}
\begin{enumerate}
\item Let $h\in H,k\in K$. Note $hk=kh \Leftrightarrow hkh^{-1}k^{-1}=1$. Since $H\unlhd G,\ kh^{-1}k^{-1}\in H$, so $hkh^{-1}k^{-1}\in H$. By symmetry of $H$ and $K,\ hkh^{-1}k^{-1}\in K$ as well, so $hkh^{-1}k^{-1}=1$ as desired.
\item Define $\varphi:H\times K\rightarrow HK=G$ by $(h,k)\mapsto hk$. Sanity check: if $(h_1,k_1),(h_2,k_2)\in H\times K$ then $\varphi((h_1,k_1)(h_2,k_2))=\varphi((h_1h_2,k_1k_2))=h_1h_2k_1k_2=h_1k_1h_2k_2=\varphi((h_1,k_1))\varphi((h_2,k_2))$. It immediately follows from assumption that $\varphi$ is surjective. Now if $(h,k)\in\ker\varphi$ then $h=k^{-1}\in H\cap K=\{1\}$, so $h=k=1$ and $hk=1$, i.e. $\ker\varphi=\{1\}$ which implies $\varphi$ is injective. 
\end{enumerate}
\end{proof}

\begin{remark}
The hypotheses of this lemma are not too bad to work with. Lagrange's theorem allows us to study $H\cap K$, Proposition \ref{prop:orderofHK} allows to study $HK$, and Sylow theorems say a lot about normality.
\end{remark}

\begin{defn}
An isomorphism $\phi:G\rightarrow G$ is an \textit{automorphism} of $G$. The set $\Aut(G):=\{\phi:\phi\text{ an automorphism}\}$ is a group under composition, called the \textit{automorphism group} of $G$.
\end{defn}
\begin{example}
\label{example:auto}
\begin{enumerate}
\item $\id:G\rightarrow G$ is an automorphism.
\item If $G=C_p$ where $p$ is prime, then $f_e:G\rightarrow G:x_i\mapsto x^{ie}\in \Aut(G)$ for $1\leq e\leq p-1$. Furthermore, this is in fact all the automorphisms and $\Aut(C_p)\cong C_{p-1}$ (see sheet 4 Q8).
\item If $K\unlhd G$ and $g\in G$, then $c_g:K\rightarrow K:x\mapsto gxg^{-1}\in\Aut(K)$.
\end{enumerate}
\end{example}

\begin{defn}
Let $H,K$ be groups and $\phi:H\rightarrow \Aut(K)$ a homomorphism. For $h\in H$, write $\phi_h$ in place of $\phi(h)$. Define a binary operation $\ast:(H\times K)\times(H\times K)\rightarrow H\times K$ by $(h_1,k_1)\ast (h_2,k_2) = \left(h_1h_2,\phi_{h_2^{-1}} (k_1)k_2\right)$. Then $(H\times K,\ast)$ is a group, called the \textit{semidirect product} of $H$ and $K$ with respect to $\phi$, denoted $H\ltimes_\phi K$.
\end{defn}

\begin{remark}[Defence of the definition]
This is not as weird as it looks. If $x,y\in G$ then $xy=yc_{y^{-1}}(x)$ where $c_y$ is as in Example \ref{example:auto}.3 above. Also, this really is a generalisation of the direct product. To see this, define $\phi_h$ to be $\id_K \ \forall h\in H$.
\end{remark}

\begin{flushright}
\textit{Week 7, lecture 2 starts here}
\end{flushright}

\begin{example}
\begin{enumerate}
\item Inversion homomorphism: let $H=\langle x\rangle$ with $|x|=2$ and $K$ be abelian. Define $\phi:H\rightarrow\Aut(K)$ by $\phi_{1_H}=\id_K$ and $\phi_x(k)=k^{-1}$.

Check $\phi_x$ is an automorphism: indeed $\phi_x\in\Aut(K)$ since it's clearly bijective and as $K$ is abelian, $\phi_x(k_1k_2)=k_2^{-1}k_1^{-1}=k_1^{-1}k_2^{-1}=\phi_x(k_1)\phi_x(k_2)$.

Check $\phi$ is a homomorphism, i.e. $\phi_{h_1h_2}=\phi_{h_1}\circ\phi_{h_2} \ \forall h_1,h_2\in H$, which is not difficult to show.
\item Conjugation homomorphism: let $G$ be finite and $K\unlhd G,H\leq G$. Define $\phi$ by $\phi_h(k)=hkh^{-1}$. Again it's not difficult to do the two sanity checks.
\end{enumerate}
\end{example}

\begin{lemma}[General form of Lemma \ref{lemma:GisHKthenGcongHKifHKnormalanditsc1}]
If $H\leq G,K\unlhd G$ with $H\cap K=\{1_G\}$ and $G=HK$, then $G\cong H\ltimes_\phi K$ where $\phi$ is conjugation homomorphism.
\end{lemma}

\begin{proof}
Again it suffices to show $f:H\ltimes_\phi K\rightarrow G:f((h,k))=hk$ is an isomorphism. Let $(h_1,k_1),(h_2,k_2)\in H\ltimes_\phi K$, then
\[
\begin{aligned}
f((h_1,k_1)(h_2,k_2))&=f((h_1h_2,\phi_{h_2^{-1}}(k_1)k_2))=h_1h_2\phi_{h_2^{-1}}(k_1)k_2\\
&=h_1h_2(h_2^{-1}k_1h_2)k_2=h_1k_1h_2k_2=f((h_1,k_1))f((h_2,k_2)).
\end{aligned}
\]
To show $f$ is bijective is similar to proof of Lemma \ref{lemma:GisHKthenGcongHKifHKnormalanditsc1}.
\end{proof}

\begin{example}[Dihedral groups as semidirect products]
Recall Definition \ref{defn:dihedralgp}, write $G=D_{2n}=\langle\sigma,\tau\rangle$, and let $C_2\cong H=\langle \tau\rangle\leq G,\ C_n\cong K=\langle \sigma\rangle\unlhd G$. Proposition \ref{prop:orderofHK} says
\[
|HK|=\frac{|H||K|}{|H\cap K|}=\frac{2\times n}{1}=2n=|G|
\]
since if $H\cap K\neq\{1\}$ then it would have to be $H$ since $|H|=2$. Thus $G=HK$, so by previous lemma $G\cong H\ltimes_\phi K$ where $\phi$ is conjugation homomorphism.

Note that
\[
\phi_\tau(\sigma)=\tau\sigma\tau^{-1}=(\tau(1),\tau(2),\ldots,\tau(n))=(n,n-1,\ldots,1)=\sigma^{-1}
\]
and in general $\phi_\tau(\sigma^i)=\sigma^{-i}$, so $\phi$ is also inversion homomorphism.
\end{example}

\begin{lemma}[Generalising example above]
Let $G$ be nonabelian and finite. If \begin{itemize}
\item $G$ has a cyclic subgroup $K$ of order $\frac{|G|}{2}=:n$,
\item $G\backslash K$ has an element of order 2, and
\item the only $i\in\{1,\ldots,n-1\}:i^2\equiv 1\Mod n$ are $1,n-1$,\hfill ($\dagger$)
\end{itemize}
then $G\cong D_{2n}$.
\end{lemma}

\begin{proof}
First note that $\dagger$ is satisfied when $n=6,\ n=p$ or $n=p^2$ where $p$ is prime.
\end{proof}

\begin{flushright}
\textit{Week 7, lecture 3 starts here}
\end{flushright}

\section{Soluble group}
\end{document}